\documentclass{article} % For LaTeX2e
\usepackage{nips14submit_e,times}
\usepackage{hyperref}
\usepackage{url}
\usepackage{subfigure} 
\usepackage{graphicx}
\usepackage{wrapfig}
\usepackage{algorithm}
\usepackage{algorithmic}
\usepackage{color}
\hypersetup{ % play with the different link colors here
    colorlinks,
    citecolor=blue,
    filecolor=blue,
    linkcolor=blue,
    urlcolor=blue % set to black to prevent printing blue links
}

\usepackage{eqparbox}

\renewcommand\algorithmiccomment[1]{%
  \hfill\#\ \eqparbox{COMMENT}{#1}%
}
%\documentstyle[nips14submit_09,times,art10]{article} % For LaTeX 2.09
\newcommand{\loss} {\mathcal{L}}

\newcommand{\A}{{\mathcal A}}
\newcommand{\W}{{\mathcal W}}
\newcommand{\w}{{\mathbf w}} 
\newcommand{\X}{{\mathcal X}}
\newcommand{\Y}{{\mathcal Y}}

\newcommand{\Z}{{\mathcal Z}} 
\newcommand{\C}{{\mathcal C}} 
\newcommand{\R}{{\mathbb R}}   %% mathbb not working?
\newcommand{\F}{{\mathcal F}} 
\newcommand{\V}{{\mathcal V}} 




\usepackage{amsmath,amssymb,amsthm,bm}
\DeclareMathOperator*{\argmax}{argmax}
\DeclareMathOperator*{\argmin}{argmin}

\usepackage{accents}
\newcommand{\Wstar}{\accentset{\ast}{\W}}

\newtheorem{theorem}{Theorem}
\newtheorem{lemma}[theorem]{Lemma}
\newtheorem{definition}[theorem]{Definition}
\newtheorem{proposition}[theorem]{Proposition}
\newtheorem{assume}[theorem]{Assumption}
\newtheorem{corollary}[theorem]{Corollary}

\newcommand{\eat}[1]{}
\newcommand{\ryedit}[1]{\textcolor{blue}{\emph{[RY: #1]}}}
\newcommand{\tbedit}[1]{\textcolor{magenta}{\emph{[TB: #1]}}}


\title{Fast Multivariate Spatio-temporal Analysis\\ via Low Rank Tensor Learning}


\author{
Mohammad Taha Bahadori\thanks{ Authors have equal contributions.} \\
Dept. of Electrical Engineering\\
Univ. of Southern California\\
Los Angeles, CA 90089 \\
\texttt{mohammab@usc.edu} \\
\And
Qi (Rose) Yu$^*$ \\
Dept. of Computer Science \\
Univ. of Southern California\\
Los Angeles, CA 90089 \\
\texttt{qiyu@usc.edu} \\
\And
Yan Liu \\
Dept. of Computer Science \\
Univ. of Southern California\\
Los Angeles, CA 90089 \\
\texttt{yanliu.cs@usc.edu} 
}

% The \author macro works with any number of authors. There are two commands
% used to separate the names and addresses of multiple authors: \And and \AND.
%
% Using \And between authors leaves it to \LaTeX{} to determine where to break
% the lines. Using \AND forces a linebreak at that point. So, if \LaTeX{}
% puts 3 of 4 authors names on the first line, and the last on the second
% line, try using \AND instead of \And before the third author name.

\newcommand{\fix}{\marginpar{FIX}}
\newcommand{\new}{\marginpar{NEW}}

\nipsfinalcopy % Uncomment for camera-ready version

\begin{document}


\maketitle

\begin{abstract}
Accurate and efficient analysis of multivariate spatio-temporal data is critical in climatology, geology, and sociology applications. Existing models usually assume simple inter-dependence among variables, space, and time, and are computationally expensive. We propose a unified low rank tensor learning framework for multivariate spatio-temporal analysis, which can conveniently incorporate different properties in spatio-temporal data, such as spatial clustering and feature sharing.   We demonstrate how the general framework can be applied to cokriging and forecasting tasks, and develop an efficient greedy algorithm to solve the resulting optimization problem with convergence guarantee. We conduct experiments on both synthetic datasets and real application datasets to demonstrate that our method is not only significantly faster than existing methods but also achieves lower estimation error. 

\end{abstract}

\section{Introduction}
%introduction
Spatio-temporal data provide unique information regarding ``where'' and ``when'', which is essential to answer many important  questions in scientific studies from geology, climatology to sociology. From a machine learning perspective, we are confronted with a series of new challenges when analyzing spatio-temporal data because of the complex spatial and temporal dependencies involved. 

A plethora of excellent work has been conducted to address the challenge and achieved successes to a certain extent, see e.g. \cite{cressie2010fixed, isaaks2011applied} and the references therein. Often times, geostatistical models use cross variogram and cross covariance functions to describe the intrinsic dependency structure. However, the parametric form of the cross variogram and the cross covariance functions impose strong assumptions on the correlation among the data. Moreover, choosing a valid variogram function requires domain knowledge and manual work. Most importantly, parameter learning of those statistical models is computationally expensive, making them infeasible for  large-scale spatio-temporal applications. 

Cokriging and forecasting are the two central tasks in multivariate spatio-temporal analysis. Cokriging utilizes the spatial correlations to predict the value of the variables for new locations. One widely adopted method is the  Multitask Gaussian Process (MTGP) \cite{bonilla2007multi}, which assumes the Gaussian process prior on the observations. However, for a cokriging task with $M$ variables of $P$ locations for $T$ time stamps, the time complexity of MTGP is $\mathcal{O}(M^3P^3T)$. 
%Acceleration techniques such as Nystr{\"o}m method and incomplete-Cholesky decomposition are employed, but they do not fully capture the correlations. 
For forecasting,  popular methods in multivariate time series analysis include the vector autoregressive (VAR) model, the autoregressive integrated moving average (ARIMA) model, and cointegration models. Bayesian hierarchical spatio-temporal models have studied both space-time separable and non-separable covariance functions \cite{cressie1999classes}. Rank reduced models have been proposed to incorporate the interrelationship among variables \cite{anderson1951estimating}. However, none of the existing work handles the commonalities among variables, space and time simultaneously in a scalable way. In this paper, we aim to address this problem by presenting a unified framework for many spatio-temporal analysis tasks that are scalable for large-scale applications. % without assuming the explicit form of the commonalities among those dimensions. 

%Since the multivariate spatio-temporal data come in the form of (variable $\times$ time $\times$ location), it is natural to represent it using tensors. 
Tensor representation provides a convenient way to incorporate the inter-dependencies along multiple dimensions. Therefore it is natural to represent the multivariate spatio-temporal data  in tensor.  Recent advances in low rank learning have led to simple models that can  capture the commonalities among each mode of the tensor and produce simpler models, which are easier to learn. Similar assumptions can be seen in the literature of spatial data recovery \cite{gandy2011tensor}, neuroimaging analysis \cite{zhou2013tensor}, and  multi-task learning \cite{romera2013multilinear}. Our work builds upon the recent advances in low rank tensor learning \cite{kolda2009tensor, gandy2011tensor, zhou2013tensor} and further considers the scenario where additional side information for the data is available. For the geo-spatial applications, apart from measurements of multiple variables, we also utilize the geographical information. For social network applications, we take advantage of the friendship network structure.  To utilize the side information, we construct similarity kernel based on them and regularize with the corresponding Laplacian matrix, which favors locally smooth solutions.

We develop a fast greedy algorithm for learning low rank tensors based on the greedy structure learning framework \cite{Barron2008,Zhang2011,Shwartz11}.  
The greedy low rank tensor learning is efficient, as it does not require full singular value decomposition of large matrices as opposed to other alternating direction methods. 
We also provide a bound on the difference between the loss function at the greedy algorithm solution and  the globally optimal solution. %\ryedit{Singular value statement is too technical for introduction. What is the condition for global optimality.}
 Finally, we present simulation results as well as the empirical evaluation results on climate and social network data where our algorithm achieves higher accuracy with increased speed than state-of-art approaches in cokriging and forecasting tasks.

%\ryedit{We don't need separate notation paragraph, introduce when first appear}
%\paragraph{Notations} In this paper, we use lowercase letter for scalers, bold font small letter for vectors, capital letter for matrices and calligraphic font for tensors. We describe the tensor basics and notations to be used in this paper. 
% We use three types of matrix norms: For any matrix $A \in \mathbb{R}^{p\times q}$ with singular values $\sigma_1 \geq \sigma_2 \geq \ldots \sigma_{\min(p, q)}\geq 0$, the \textit{Frobenius norm} is defined as $\|A\|_F = \sqrt{\sum_{i=1}^{\min(p, q)} \sigma_i^2 }$ and the \textit{Nuclear norm} (also called \textit{Trace norm}) is defined as $\|A\|_* = \sum_{i=1}^{\min(p, q)} \sigma_i$ and the \textit{Operator norm} is defined as $\|A\|_{op} = \|A\|_2 = \sigma_1$. We also use the colon operator : to select a slice of a tensor similar to MATLAB's syntax.


\vspace{-0.1in}
\section{Tensor formulation for multivariate spatio-temporal analysis}
\vspace{-0.1in}
The critical element in multivariate spatio-temporal analysis is an efficient way to incorporate the spatial temporal correlations  into the model and automatically capture the shared structures among variables, locations, and time. In this section, we present a unified low rank tensor learning framework that can perform various types of spatio-temporal analysis. We will use two important applications, i.e., cokriging and forecasting,  to motivate and describe the framework. %,  which not only captures the complex correlations in the data, but also is scalable. 
\subsection{Cokriging}
% cokriging description
In geostatistics, cokriging is the task of interpolating the data of one variable for unknown locations by taking advantage of the observations of variables from known locations. For example, by making use of the correlations between precipitation and temperature, we can obtain more precise estimate of temperature in unknown locations than univariate kriging. Formally, denote the complete data for $P$ locations over $T$ time stamps with $M$ variables as $\X \in \mathbb{R}^{P\times T\times M} $. We only observe the measurements for a subset of locations $\Omega \subset \{1, \ldots, P\}$ and their side information such as longitude and latitude. Given the measurements $\X_{\Omega}$ and the side information, the goal is to estimate a tensor $\W \in \mathbb{R}^{P\times T\times M}$ that satisfies $\W_\Omega =  \X_\Omega$. Here $\X_\Omega$ represents the outcome of applying the index operator $I_\Omega$ to $\X_{:, :, m}$ for all variables $ m= 1, \ldots, M$. The index operator $I_\Omega$ is a diagonal matrix whose entries are one for the locations included in $\Omega$ and zero otherwise.


Existing work have identified two key consistency principles for performing cokriging \cite[Chapter 6.2]{cressie2011statistics}: (1) Global consistency: the data in the common structure (variable, space, time) are likely to be similar. (2) Local consistency: the data in close locations are likely to be similar. These principles are akin to the \textit{cluster assumption} for semi-supervised learning \cite{zhou2003learning}. We incorporate these principles in a concise and computationally efficient low-rank tensor learning framework.

To achieve global consistency, we constrain the tensor $\W$ to be low rank. The low rank assumption is based on the belief that high correlations exist within variables, locations and time, which leads to natural clustering of the data. Existing literature have explored the low rank structure among these three dimensions separately, e.g., multi-task learning \cite{nie2010efficient} for variable correlation, fixed rank kriging \cite{cressie2008fixed} for spatial correlations. Low rankness assumes that the observed data can be described with a few latent factors. It enforces the commonalities along three dimensions without an explicit form for the shared structures in each dimension.

For the local consistency, we construct a regularizer via the spatial Laplacian matrix. The Laplacian matrix is defined as  $L = D-A$, where $A$ is a kernel matrix constructed by pairwise similarity %$A_{i,j} = \mathrm{similarity}(\mathrm{location}(i), \mathrm{location}(j))$
and diagonal matrix $D_{i,i}= \sum_{j} (A_{i,j})$. %For every vector $\mathbf{x}$, the Laplacian regularizer is applied as $\mathrm{tr}(\mathbf{x}^{\top}L\mathbf{x}) = \sum_{i, j}A_{i,j}(x_i-x_j)^2$ which smoothes the entries of $\mathbf{x}$ that are similar according to $A$; 
Similar ideas have been used in matrix completion \cite{li2009relation}. In cokriging literature, the local consistency is enforced via the spatial covariance matrix. The Bayesian models often impose the Gaussian process prior on the observations with the covariance matrix $K = K_v \otimes K_x $ where $K_v$ is the covariance between variables and $K_x$ is that for locations. The Laplacian regularization term corresponds to the relational Gaussian process \cite{chu2006relational} where the covariance matrix is approximated by the spatial Laplacian.


In summary, we can perform cokriging and find the value of tensor $\W$ by solving the following optimization problem:
\begin{align}
\widehat{\W} =\argmin_{\W}& \left\{ \|\W_\Omega - \X_\Omega \|^2_F +  \mu \sum\limits_{m=1}^M \text{tr} (\W_{:,:,m}^\top L \W_{:,:,m}) \right\} \quad
\mathrm{s.t.} \;&  \text{rank}(\W) \leq \rho,  \label{eqn:cokriging}
\end{align}
\noindent  where the Frobenius norm of a tensor $\A$ is defined as $\|\A\|_F = \sqrt{\sum_{i, j, k}\A_{i, j, k}^2}$ and $\mu, \rho > 0$ are the parameters that make  tradeoff between the local and global consistency, respectively.  The low rank constraint finds the principal components of the tensor and reduces the complexity of the model while the Laplacian regularizer clusters the data using the relational information among the locations.  By learning the right tradeoff between these two techniques, our method is able to benefit from both of them. Due to the various definitions of tensor rank, we use \textit{rank} as supposition for rank complexity, which will be specified in later section.

\subsection{Forecasting}
% multi-task forecasting
Forecasting estimates the future value of the multivariate time series given the historical observations. Simply, we use the classical VAR model with  $K$ lags and coefficient tensor $\W \in \mathbb{R}^{P\times LP\times M}$. Using the matrix representation, the VAR($K$) process defines the following data generation process:
\begin{equation}
\X_{:,t,m} = \mathcal{W}_{:,:, m}\mathbf{X}_{t, m} + \mathcal{E}_{:,t,m}, \quad \text{for } m = 1, \ldots, M \text{ and } t = K+1, \ldots, T,
\label{eqn:auto-regressive} 
\end{equation}
\noindent where $\mathbf{X}_{t, m} = [\X_{:, t-1, m}^{\top}, \ldots, \X_{:, t-K, m}^{\top}]^{\top}$ denotes the concatenation of $K$-lag historical data before time $t$. The noise tensor $\mathcal{E}$ is a multivariate Gaussian with zero mean and unit variance .

Existing multivariate regression methods designed to capture the complex correlations, such as Tucker decomposition \cite{romera2013multilinear}, are computationally expensive. 
A scalable solution requires a simpler model that also efficiently accounts for the shared structures in variables, space, and time. Similar global and local consistency principles still hold in forecasting. For global consistency, we can use low rank constraint to capture the commonalities of the variables as well as the spatial correlations on the model parameter tensor, as in \cite{cressie2010fixed}. For local consistency, we enforce the predicted value for close locations to be similar via spatial Laplacian regularization. Thus, we can formulate the forecasting task as the following optimization problem over the model coefficient tensor $\W$:
\begin{equation}
\widehat{\W} = \argmin_{\W} \left\{ \| \widehat{\X}- \X \|^2_F +  \mu \sum\limits_{m=1}^M \text{tr} (\widehat{\X}_{:,:,m}^\top L \widehat{\X}_{:,:,m}) \right\} 
\;\;\text{s.t.} \;\; \text{rank}(\W) \leq \rho, \; \widehat{\X}_{:,t,m} = \mathcal{W}_{:,:, m}\mathbf{X}_{t, m}
\label{eqn:forecasting}
\end{equation}

Though cokriging and forecasting are two different tasks, we can easily see that both formulations follow the global and local consistency principles and can capture the inter-correlations from spatial-temporal data.

 
\subsection{Unified Framework}
% unified formulation
We now show that both cokriging and forecasting can be formulated into the same tensor learning framework. Let us rewrite the loss function in Eq. (\ref{eqn:cokriging}) and Eq. (\ref{eqn:forecasting}) in the form of multitask regression and complete the quadratic form for the loss function. The cokriging task can be reformulated as follows:
\begin{align}
\widehat{\W} = \argmin_{\W}& \left\{\sum_{m=1}^{M} \|\W_{:, :, m}H - (H^{\top})^{-1}\X_{\Omega, m} \|^2_F\right\} \quad
\mathrm{s.t.} \quad \text{rank}(\W) \leq \rho  \label{eqn:cokriging_reformulate}
\end{align}
\noindent where we define $HH^{\top} = I_{\Omega}+\mu L$.\footnote{We can use Cholesky decomposition to obtain $H$. In the rare cases that $I_{\Omega}+\mu L$ is not full rank, $\epsilon I_{P}$ is added where $\epsilon$ is a very small positive value.} For the forecasting problem, $HH^{\top} = I_{P}+\mu L$ and we have:
\begin{align}
\widehat{\W} = \argmin_{\W} \left\{\sum_{m=1}^{M}\sum_{t=K+1}^{T} \|H\W_{:, :, m}\mathbf{X}_{t, m} - (H^{-1})\X_{:, t, m} \|^2_F\right\}
\quad \mathrm{s.t.}  \quad\text{rank}(\W) \leq \rho,  \label{eqn:forecasting_reformulate}
\end{align}
By slight change of notation (cf. Appendix \ref{sec:derive}), we can easily see that the optimal solution of both problems can be obtained by the following optimization problem with appropriate choice of tensors $\Y$ and $\V$:
\begin{equation}
\widehat{\W} = \argmin_{\W} \left\{\sum_{m=1}^{M} \|\W_{:, :, m}\Y_{:, :, m} - \V_{:, :, m} \|^2_F\right\} \quad
\mathrm{s.t.} \quad\text{rank}(\W) \leq \rho.
\label{eq:greedyUnified}
\end{equation}

After unifying the objective function, we note that tensor rank has different notions such as CP rank, Tucker rank and mode n-rank \cite{kolda2009tensor,gandy2011tensor}. In this paper,  we choose the mode-n rank, which is computationally more tractable \cite{gandy2011tensor,tomioka2010estimation}. The mode-n rank of a tensor $\W$ is the rank of its mode-n unfolding $\W_{(n)}$.\footnote{The mode-$n$ unfolding of a tensor is the matrix resulting from treating $n$ as the first mode of the matrix, and cyclically concatenating other modes. Tensor refolding is the reverse direction operation \cite{kolda2009tensor}.} In particular, for a tensor $\W$ with $N$ mode, we have the following definition:
\begin{equation}
\text{mode-n rank}(\W) = \sum\limits_{n=1}^N \text{rank}(\W_{(n)}).
\end{equation}
 A common practice to solve this formulation with mode $n$-rank constraint  is to relax the rank constraint to a convex nuclear norm constraint \cite{gandy2011tensor,tomioka2010estimation}. However, those methods are  computationally expensive since they need full singular value decomposition of large matrices. In the next section, we present a fast greedy algorithm to tackle the problem.



\section{Fast greedy low rank tensor learning}


%\subsection{Non-convex greedy algorithm}
% greedy
To solve the non-convex problem in Eq. (\ref{eq:greedyUnified}) and find its optimal solution, we propose a greedy learning algorithm by successively adding rank-1 estimation of the mode-n unfolding. The main idea of the algorithm is to unfold the tensor into a matrix,  seek for its rank-1 approximation and then fold back into a tensor with same dimensionality. We describe this algorithm in three steps: (i) First, we show that we can learn rank-1 matrix estimations efficiently by solving a generalized eigenvalue problem, (ii) We use the rank-1 matrix estimation to greedily solve the original tensor rank constrained problem, and (iii) We propose an enhancement by orthogonal projections after each greedy step.

\paragraph{Optimal rank-1 Matrix Learning} %After we have a unified formulation of our learning tasks in Eq. (\ref{eq:greedyUnified}), we specify the greedy updates for solving it. 
The following lemma enables us to find such optimal rank-1 estimation of the matrices.

\begin{lemma}
\label{lem:rank1opt}
%Suppose $n$ pairs of observations $(\mathbf{x}_i, \mathbf{y}_i)\in {\R}^{p\times 1}\times {\R}^{q\times 1}$ are given. Consider the following rank-1 estimation problem:
%\begin{align}
%\widehat{A}_1 &= \argmin_{A, \mathrm{rank}(A) = 1}\left\{\frac{1}{n}\sum_{i=1}^{n}\|\mathbf{y}_i - A\mathbf{x}_i\|_2^2 \right\}\label{eq:rank1estimate}
%\end{align}
Consider the following rank constrained problem:
\begin{equation}
\widehat{A}_1 = \argmin_{A: \mathrm{rank}(A) = 1} \left\{ \left\| Y - AX\right\|_F^2 \right\},
\label{eq:rank1estimate}
\end{equation}
\noindent where $Y \in {\R}^{q\times n}$, $X \in {\R}^{p\times n}$, and $A\in \mathbb{R}^{q\times p}$. The optimal solution of $\widehat{A}_1$ can be written as $\widehat{A}_1 = \widehat{\mathbf{u}}\widehat{\mathbf{v}}^{\top}$, $\|\widehat{\mathbf{v}}\|_2 = 1$ where $\widehat{\mathbf{v}}$ is the dominant eigenvector of the following generalized eigenvalue problem:
\begin{equation}
(XY^{\top}YX^{\top})\mathbf{v} = \lambda (XX^{\top})\mathbf{v}\label{eq:solv}\\
\end{equation}
and $\widehat{\mathbf{u}}$ can be computed as
\begin{align}
&\widehat{\mathbf{u}} = \frac{1}{\widehat{\mathbf{v}}^{\top}XX^{\top}\widehat{\mathbf{v}}}YX^{\top}\widehat{\mathbf{v}}. \label{eq:solu}
\end{align}
\end{lemma}

Proof is deferred to the Appendix \ref{sec:optRank1}. Eq. (\ref{eq:solv}) is a generalized eigenvalue problem whose dominant eigenvector can be found efficiently \cite{jpen2000}. If $XX^{\top}$ is full rank, as we will assume in Theorem \ref{thm:greedy}, the problem is simplified to the regular eigenvalue problem whose dominant eigenvector can be efficiently computed.
%which can be solved in $\mathcal{O}(n_z\log(p))$ iterations if the matrix has $n_z$ non-zero elements \cite{Kuczynski1992}.  

\paragraph{Greedy Low n-rank Tensor Learning} The optimal rank-1 matrix learning serves as a basic element in our greedy algorithm. 
Using Lemma \ref{lem:rank1opt}, we can solve the problem in Eq. (\ref{eq:greedyUnified}) in the \textit{Forward Greedy Selection} framework as follows:  at each iteration of the greedy algorithm, it searches for the mode that gives the largest decrease in the objective function. It does so by unfolding the tensor in that mode and finding the best rank-1 estimation of the unfolded tensor. After finding the optimal mode, it adds the rank-1 estimate in that mode to the current estimation of the tensor. Denoting $\loss(\W;\Y, \V ) = \sum_{m=1}^{M} \|\W_{:, :, m}\Y_{:, :, m} - \V_{:, :, m} \|^2_F$, Algo. \ref{alg:greedy} shows the details of this approach.  Note that we can find the optimal rank-1 solution in only one of the modes, but it is enough to guarantee the convergence of our greedy algorithm. %rate in Theorem \ref{thm:greedy}.
% Due to the page limit, a detailed example of sub-problem in each fold is described in Appendix \ref{sec:gfolding}.

\begin{algorithm}[t]
 \caption{Greedy Low-rank Tensor Learning }
 \label{alg:greedy}
\begin{algorithmic}[1]
   \STATE {\bfseries Input:}
   transformed data $\Y, \V$ of $M$ variables, stopping criteria $\eta$
   \STATE {\bfseries Output:} $N$ mode tensor $\W$ 
  \STATE  Initialize $\W \gets 0$
  \REPEAT 
    \FOR{$n=1$ {\bfseries to} $N$}
    \STATE $B_n \leftarrow  \argmin\limits_{B:~\mathrm{rank}(B) = 1}\loss(\mathrm{refold}(\W_{(n)}+B); \Y, \V)$
 	\STATE  
$\Delta_n\gets \loss(\W;\Y, \V ) -  \loss(\mathrm{refold}(\W_{(n)}+B_n);\Y, \V )$
	\ENDFOR
	\STATE $ n^{*}\leftarrow \argmax\limits_{n} \{\Delta_n\}$

 \IF{$ \Delta_{n^{*}}> \eta$}
% \STATE 
% $B_{n^{*}} \leftarrow  \argmin\limits_{B:\mathrm{rank}(B) = 1}\loss(\mathcal{\W}_{(n^{*})}; \X,\Y) + B)$
 \STATE
 $\W \gets \W + \mathrm{refold}(B_{n^{*}},n^{*} )$
 \ENDIF
 \STATE $\W \gets \argmin_{\begin{subarray}{c}\mathrm{row}(\A_{(1)}) \subseteq \mathrm{row}(\W_{(1)})\\ \mathrm{col}(\A_{(1)}) \subseteq \mathrm{col}(\W_{(1)})\end{subarray} } \loss(\A;\Y, \V ) $ \COMMENT{\textit{Optional Orthogonal Projection Step.}}
 \UNTIL{$\Delta_{n^{*}} < \eta$}
\end{algorithmic}
\end{algorithm}

\eat{
\begin{algorithm}[htbp]
 \caption{The Orthogonal Projection Steps at $k^{th}$ Iteration of Algorithm \ref{alg:greedy} after line 11.}
 \label{alg:orthogonal}
\begin{algorithmic}
\STATE $[U, S, V] \gets \mathrm{svd}(\mathcal{W}_{(1)}, k)$.
\STATE $\widehat{S} \gets \min_{S}\loss(USV^{\top}, \mathcal{X}, \mathcal{Y})$.
\STATE $\W \gets \mathrm{fold}(U\widehat{S}V^{\top}, 1)$s
\end{algorithmic}
\end{algorithm}}

Theorem \ref{thm:greedy} bounds the difference between the loss function evaluated at each iteration of the greedy algorithm and the globally optimal solution.

\begin{theorem}\label{thm:greedy}
Suppose in Eq. (\ref{eq:greedyUnified}) the matrices $\Y_{:, :, m}^{\top}\Y_{:, :, m}$ for $m = 1, \ldots, M$ are positive definite.  The solution of Algo. \ref{alg:greedy} at its $k$th iteration step satisfies the following inequality:
\begin{equation}
\mathcal{L}(\W_{k};\Y,\V) - \mathcal{L}(\W^{*};\Y,\V) \leq \frac{ (\| \Y \|_{2}\|\W_{(1)}^*\|_{*})^2 }{(k+1)},
\end{equation}
\noindent where $\W^*$ is the global minimizer of the problem in Eq. (\ref{eq:greedyUnified}) and $\| \Y\|_{2}$ is the largest singular value of a block diagonal matrix created by placing the matrices $\Y(:, :, m)$ on its diagonal blocks.
\end{theorem}

The detailed proof is given in Appendix \ref{sec:gProof}. The key idea of the proof is that the amount of decrease in the loss function by each step in the selected mode is not smaller than the amount of decrease if we had selected the first mode. The theorem shows that we can obtain the same rate of convergence for learning low rank tensors as achieved in \cite{ShalevShwartz2010} for learning low rank matrices. The greedy algorithm in Algo. \ref{alg:greedy} is also connected to the mixture regularization in \cite{tomioka2010estimation}: the mixture approach decomposes the solution into a set of low rank structures while the greedy algorithm successively learns a set of rank one components. 

\paragraph{Greedy Algorithm with Orthogonal Projections} It is well-known that the forward greedy algorithm may make steps in sub-optimal directions because of noise.  A common solution to alleviate the effect of noise is to make orthogonal projections after each greedy step, see for e.g. \cite{Barron2008,Shwartz11}. Thus, we enhance the forward greedy algorithm by projecting the solution into the space spanned by the singular vectors of its mode-1 unfolding. The  greedy algorithm with \textit{orthogonal} projections performs an extra step in line 13 of Algo. \ref{alg:greedy}:  It finds the top $k$ singular vectors of the solution: $[U, S, V] \gets \mathrm{svd}(\mathcal{W}_{(1)}, k)$ where $k$ is the iteration number. Then it finds the best solution in the space spanned by $U$ and $V$ by solving $\widehat{S} \gets \min_{S}\loss(USV^{\top}, \Y, \V)$ which has a closed form solution. Finally, it reconstructs the solution: $\W \gets \mathrm{refold}(U\widehat{S}V^{\top}, 1)$. Note that the projection only needs to find top $k$ singular vectors which can be computed efficiently for small values of $k$.




%In order to describe our greedy algorithm for mixture low-rank tensor learning, we first describe it in a simpler multi-task linear regression setting; then we show how we can solve the target problems. Consider the problem of finding a tensor $\mathcal{A}\in \mathbb{R}^{q\times p\times r}$  in a multisource regression problem of predicting $\mathbf{y}^{(j)}\in \mathbb{R}^{q\times 1}$ using $\mathbf{x}^{(j)}\in \mathbb{R}^{p\times 1}$ for tasks $j = 1, \ldots, r$ based on the following model:
%\begin{equation}
%\mathbf{y}^{(j)} = \mathcal{A}(:, :, j)\mathbf{x}^{(j)} + \bm{\varepsilon}^{(j)}
%\end{equation}
%\noindent for $j = 1, \ldots, r$.  Suppose we are given $n$ pairs of $(\mathbf{x}^{(j)}, \mathbf{y}^{(j)})$ for each task.  If we stack the observations as $Y^{(j)} \in \mathbb{R}^{q\times n}$ and $X^{(j)} \in \mathbb{R}^{p\times n}$ we can write:
%\begin{equation}
%\widehat{\mathcal{A}} = \argmin_{\mathcal{A}}\left\{\frac{1}{n}\sum_{j=1}^{r}\left\|Y^{(j)} - \mathcal{A}(:, :, j)X^{(j)} \right\|_F^2 \right\}
%\end{equation}


\vspace{-0.1in}
\section{Experiments}
\vspace{-0.1in}
\label{sec:exp}
% exps
We evaluate the efficacy of our algorithms on synthetic datasets and real-world application datasets.
\subsection{Low rank tensor learning on synthetic data} 
% synthetic
%As shown in Eq. (\ref{eq:greedyUnified}),
%We formulate the cokriging and the forecasting task as low rank tensor learning problems to capture the shared structure among variables, space and time. 
%To validate that our algorithms can accurately and efficiently estimate the low rank tensor, we experiment with the multitask regression model.
For empirical evaluation, we compare our method with  multitask learning (MTL) algorithms, which also utilize the commonalities between different prediction tasks for better performance. We use the following baselines: (1) Trace norm regularized MTL (\textit{Trace}), which seeks the low rank structure only on the task dimension; (2) Multilinear MTL \cite{romera2013multilinear}, which adapts the convex relaxation of low rank tensor learning solved with Alternating Direction Methods of Multiplier  (\textit{ADMM}) \cite{gabay1976dual} and Tucker decomposition to describe the low rankness in multiple dimensions; (3) \textit{MTL-$L_1$} , \textit{MTL-$L_{21}$} \cite{nie2010efficient}, and\textit{ MTL-$L_{\mathrm{Dirty}}$}  \cite{jalali2010dirty}, which investigate joint sparsity of the tasks with $L_p$ norm regularization. For MTL-$L_1$ , MTL-$L_{21}$ \cite{nie2010efficient} and MTL-$L_{\mathrm{Dirty}}$, we use MALSAR Version 1.1 \cite{zhou2012mutal}. 


We construct a model coefficient tensor $\W$ of size $20 \times 20 \times 10 $ with CP rank equals to $1$.  Then, we generate the observations $\Y$ and $\V$ according to multivariate regression model $\V_{:,:,m} = \W_{:,:,m}\Y_{:, :, m} + \mathcal{E}_{:,:,m}$ for $m = 1, \ldots, M$, where $\mathcal{E}$ is tensor with zero mean Gaussian noise elements. We split the synthesized data into training and testing time series and vary the length of the training time series from $10$ to $200$. For each training length setting, we repeat the experiments for $10$ times and select the model parameters via 5-fold cross validation. We measure the prediction performance  via two criteria: parameter estimation accuracy and rank complexity. For accuracy, we calculate the RMSE of the estimation versus the true model coefficient tensor. For rank complexity, we calculate the mixture rank complexity \cite{tomioka2010estimation} as $MRC = \frac{1}{n}\sum_{n=1}^{N}\mathrm{rank}(\W_{(n)})$. 

\begin{figure*}[t]
%\vskip 0.2in
\centering 
\subfigure[RMSE]{
\includegraphics[scale = 0.18]{figures/RMSE_est_Synth.pdf}
\label{fig:RMSE_est}
}
\subfigure[Rank]{
\includegraphics[scale = 0.18]{figures/LR_complexity.pdf}
\label{fig:LR_complexity}
}
\subfigure[Scalability]{
\includegraphics[scale = 0.18]{figures/Scalability_Synth.pdf}
\label{fig:Scalability}
}
\label{fig:synthetic}
\caption{Tensor estimation performance comparison on the synthetic dataset over 10 random runs.  \subref{fig:RMSE_est} parameter Estimation RMSE with training time series length,  \subref{fig:LR_complexity} Mixture Rank Complexity with training  time series length,  \subref{fig:Scalability} running time for one single round with respect to number of variables.}
\vskip -0.2in
\end{figure*} 


The results are shown in Figure \ref{fig:RMSE_est} and \ref{fig:LR_complexity}. We omit the Tucker decomposition as the results are not comparable. We can clearly see that the proposed greedy algorithm with orthogonal projections achieves the most accurate tensor estimation. In terms of rank complexity, we make two observations: (i) Given that the tensor CP rank is 1, greedy algorithm with orthogonal projections produces the estimate with the lowest rank complexity. This can be attributed to the fact that the orthogonal projections eliminate the redundant rank-1 components that fall in the same spanned space. (ii) The rank complexity of the forward greedy algorithm increases as we enlarge the sample size. We believe that  when there is a limited number of observations, most of the new rank-1 elements added to the estimate are not accurate and the cross-validation steps prevent them from being added to the model. However, as the sample size grows, the rank-1 estimates become more accurate and they are preserved during the cross-validation.

To showcase the scalability of our algorithm, we vary the number of variables and generate a series of tensor $\W \in \mathbb{R}^{20\times 20 \times M}$ for M from 10 to 100 and record the running time (in seconds) for three tensor learning algorithms, i.e, forward greedy, greedy with orthogonal projections and ADMM. We measure the run time on a machine with a 6-core 12-thread Intel Xenon 2.67GHz processor and  12GB memory. The results are shown in 
Figure \ref{fig:Scalability}. The running time of ADMM increase rapidly with the data size while the greedy algorithm stays steady,  which confirms the speedup advantage of the greedy algorithm. 

\vspace{-0.1in}
\subsection{Spatio-temporal analysis on real world data}
We conduct cokriging and forecasting experiments on three real-world datasets: 
\vspace{-3mm}
\paragraph{USHCN}
The U.S. Historical  Climatology Network Monthly (USHCN) \footnote{\url{ http://www.ncdc.noaa.gov/oa/climate/research/ushcn}} dataset consists of monthly climatological data of 108 stations spanning from year 1915 to 2000. It has three climate variables: (1) daily maximum, (2) minimum temperature  averaged over month, and (3) total monthly precipitation. 
\vspace{-3mm}
\paragraph{CCDS}
The Comprehensive Climate Dataset (CCDS) is a collection of climate records of North America from \cite{lozano2009spatial}. The dataset was collected and pre-processed by five federal agencies. %:  CRU (\url{http://www.cru.uea.ac.uk/cru/data }), NOAA (\url{ http://www.cdc.noaa.gov/data/gridded/}), NASA (\url { http://iridl.ldeo.columbia.edu/ SOURCES/.NASA/ .GSFC/.TOMS/.NIMBUS7/ }), NCDC (\url{http://rredc.nrel.gov/solar/old_data/nsrdb/ }) and CDIAI (\url{http://cdiac.ornl.gov/epubs/ndp/ushcn/ usa.html)}) 
It contains monthly observations of 17 variables such as Carbon dioxide and temperature spanning from 1990 to 2001. The observations were interpolated on a $2.5 \times 2.5$ degree grid, with 125 observation locations.
\vspace{-3mm}
\paragraph{Foursquare}
The Foursquare dataset \cite{long2012exploring} contains the users' check-in records in Pittsburgh area from Feb 24 to May 23, 2012, categorized by different venue types such as Art \& Entertainment, College \& University, and Food. The check-in tensor is created by counting the number of check-ins by 121 users  in each of the 15 category of venues during 1200 time intervals.
%We select a subset of dataset with 121 active users for 3767 time stamps of 15 venue category variables. 

\begin{table*}[t]
\caption{Cokriging NRMSE of 6 methods averaged over 10 runs. In each run, 10\% of the locations are assumed missing. } %title of the table
\label{tab:cokrig_RMSE}
%\vskip 0.15in
\begin{center}
\begin{tiny}
\begin{sc}
\centering  \footnotesize% centering table
\begin{tabular}{c c c c c c c c} % creating eight columns
%\abovespace\belowspace
\hline
Dataset & ADMM & Forward & Orthogonal  & Simple& Ordinary& MTGP \\
\hline
USHCN  &  0.8051 & 0.7594 & \textbf{0.7210}&  0.8760& 0.7803 & 1.0007 \\
CCDS & 0.8292 & 0.5555& \textbf{0.4532} & 0.7634 & 0.7312 & 1.0296 \\
Foursquare & 0.1373 & 0.1338& \textbf{0.1334} & NA & NA & NA \\
\hline
\end{tabular}
\end{sc}
\end{tiny}
\end{center}
\vspace{-0.25in}
\end{table*}

% cokring
\subsubsection{Cokriging}
We compare the cokriging performance of our proposed method with the classical cokriging approaches including simple kriging and ordinary cokriging with nonbias condition \cite{isaaks2011applied} which are applied to each variables separately. We further compare with multitask Gaussian process (MTGP) \cite{bonilla2007multi} which also considers the correlation among variables. We also adapt ADMM for solving the nuclear norm relaxed formulation of the cokriging formulation as a baseline (see  Appendix \ref{sec:admm_algo} for more details). For USHCN and CCDS, we construct a Laplacian matrix by calculating the pairwise Haversine distance of locations. For Foursquare, we construct the graph Laplacian from the user friendship network.

For each dataset, we first normalize it by removing the trend and diving by the standard deviation. Then we randomly pick 10\% of locations (or users for Foursquare) and eliminate the measurements of all variables over the whole time span. Then, we produce the estimates for all variables of each timestamp. We repeat the procedure for 10 times and report the normalized prediction  RMSE for all timestamps and 10 random sets of missing locations, i.e., NRMSE $=\|\widehat{\X}_{\Omega^c} - \X_{\Omega^c}\|_F/\|\X_{\Omega^c}\|_F$, where $\Omega^c$ denotes the unobserved locations. We use the  MATLAB Kriging Toolbox\footnote{\url{ http://globec.whoi.edu/software/kriging/V3/english.html}} for the classical cokriging algorithms and the MTGP code provided by \cite{bonilla2007multi}. 

Table \ref{tab:cokrig_RMSE} shows the  results for the cokriging task. The greedy algorithm with orthogonal projections is significantly more accurate in all three datasets. The baseline cokriging methods can only handle the two dimensional longitude and latitude information, thus are not applicable to the Foursquare dataset with additional friendship information. The superior performance of the greedy algorithm can be attributed to two of its properties: (1) It can obtain low rank models and achieve global consistency;  (2) It usually has lower estimation bias compared to nuclear norm relaxed methods. %During our experiments, we observe that the Laplacian  regularization with geographical information further improves the performance with the local consistency constraint.



% 0.6138 CCDS For

% forecast

\subsubsection{Forecasting}
We present the empirical evaluation on the forecasting task by comparing with multitask regression algorithms.  We split the data along the temporal dimension into 90\% training set and 10\% testing set. We choose VAR(3) model and during the training phase, we use 5-fold cross-validation.% for model selection.

\begin{table*}[t]
\caption{ Forecasting NRMSE for VAR process with 3 lags, trained with 90\%  of the time series.} %title of the table
\small
\label{tab:real_RMSE}
%\vskip 0.15in
\begin{center}
\begin{tiny}
\begin{sc}
\centering  \footnotesize% centering table
\begin{tabular}{@{}c@{\;\;} c @{\;\;} c @{\;\;}c @{\;\;}c@{\;\;} c @{\;\;}c@{\;\;} c @{\;\;}c @{\;\;}c @{\;\;}c@{}} % creating eight columns
\hline
%\abovespace\belowspace
Dataset  & Tucker  & ADMM & Forward & Ortho & OrthoNL& Trace  & MTL$_{l1}$ & MTL$_{l21}$ & MTL$_{dirty}$  \\
\hline
USHCN  & \textbf{0.8975} & 0.9227& 0.9171& 0.9069 & 0.9175 & 0.9273& 0.9528   & 0.9543 &  0.9735  \\
CCDS & 0.9438 & 0.8448 & 0.8810& \textbf{0.8325} &0.8555 &0.8632 & 0.9105& 0.9171& 1.0950 \\
FSQ  & 0.1492 & 0.1407& 0.1241& \textbf{0.1223} & 0.1234 &0.1245 &  0.1495 &  0.1495   & 0.1504  \\
%Climate 4  & & & 0.9511 & 1.1374 & 1.1374 & 0.9449  & 0.9342 & 1.1255\\
%EEG(S) &  & 0.6531  & 0.6519 &  &    &0.6132 & 0.6547& NA\\
\hline
\end{tabular}
\end{sc}
\end{tiny}
\end{center}
\vspace{-0.25in}
\end{table*}


As shown in Table \ref{tab:real_RMSE}, the greedy algorithm with orthogonal projections again achieves the best prediction accuracy. Different from the cokriging task, forecasting does not necessarily need the correlations of locations for prediction. One might raise the question as to whether the Laplacian regularizer helps. Therefore, we report the results for our formulation without Laplacian (ORTHONL) for comparison.  For efficiency, we report the running time (in seconds) in Table \ref{tab:real_runtime} for both tasks of cokriging and forecasting.  Compared with ADMM, which is a competitive baseline also capturing the commonalities among variables, space, and time, our greedy algorithm is much faster for most datasets.

\begin{table*}[t]
\caption{ Running time (in seconds)  for cokriging and forecasting.} %title of the table
\label{tab:real_runtime}
\vspace{-0.05in}
\begin{center}
\begin{tiny}
\begin{sc}
\centering  \footnotesize% centering table
\begin{tabular}{c| c c c| c c c c} % creating eight columns
%\abovespace\belowspace
\hline
&\multicolumn{3}{c}{Cokriging}& \multicolumn{3}{c}{Forecasting}\\
\hline
\hline
Dataset & USHCN  & CCDS & FSQ &  USHCN  & CCDS & FSQ \\
\hline
ORTHO  & 93.03 & 16.98& 91.51  & 75.47 & 21.38& 37.70\\
ADMM &791.25 & 320.77 & 720.40 & 235.73 &45.62 & 33.83\\
\hline
\end{tabular}
\end{sc}
\end{tiny}
\end{center}
\vspace{-0.25in}
\end{table*}

\begin{wrapfigure}{r}{0.45\textwidth}
\vspace{-0.2in}
\begin{center}
    \includegraphics[scale = 0.34]{figures/map_climate17_new.pdf}
  \end{center}
\vspace{-0.2in}
  \caption{ Map of most predictive regions analyzed by the greedy algorithm using 17 variables of  the CCDS dataset. Red color means high predictiveness whereas blue denotes low predictiveness.}
\end{wrapfigure}




%\begin{figure}[h]
%\centering
%    \includegraphics[scale = 0.35]{figures/map_climate17_new.pdf}
%  \caption{ Map of most predictive regions analyzed by the greedy algorithm using 17 variables of  the CCDS dataset. Red color means highly predictive while blue denotes low predictive.}
%  \label{fig:paramTensor}
%\end{figure}
As a qualitative study, we plot the map of most predictive regions analyzed by the greedy algorithm using CCDS dataset in Fig. 2. Based on the concept of how informative the past values of the climate measurements in a specific location are in predicting future values of other time series, we define the aggregate strength of predictiveness of each region as $w(t) = \sum_{p=1}^{P}\sum_{m=1}^{M}|\W_{p, t, m}|$. We can see that two regions are identified as the most predictive regions: (1) The southwest region, which reflects the impact of the Pacific ocean and (2) The southeast region, which frequently experiences relative sea level rise, hurricanes, and storm surge in Gulf of Mexico.  Another interesting region lies in the center of Colorado, where the Rocky mountain valleys act as a funnel for the winds from the west, providing locally divergent wind patterns.



%\begin{figure}[htbp]
%%\vskip 0.2in
%\centering 
%\includegraphics[scale = 0.3]{figures/map_climate17Color.pdf}
%\caption{ Map of most predictive regions analyzed by the orthogonal greedy algorithm using 17 agents of CCDS dataset. Red color means highly predictive while blue denotes low predictive.}
%\label{fig:paramTensor}
%\vskip -0.2in
%\end{figure} 

%\input{experiment}

%\section{Discussion}
%% discussion

\vspace{-0.1in}
\section{Conclusion}
\vspace{-0.1in}
%conclusion
In this paper, we study the problem of multivariate spatio-temporal data analysis with an emphasis on two tasks: cokriging and forecasting.  We formulate the problem into a general low rank tensor learning framework which captures both the global consistency and the local consistency of the data. We develop a fast and accurate greedy solver with theoretical guarantees for its convergence. We validate the correctness and efficiency of our proposed method on the synthetic dataset. Our extensive empirical studies on climate and social network datasets confirm the effectiveness of our framework. For future work, we are interested in investigating different forms of shared structure and extending the framework to handle non-linearity in the data.
\vspace{-0.1in}
\section*{Acknowledgment}
We thank the anonymous reviewers for their helpful feedback and comments. The research was sponsored by the NSF research grants IIS-1134990, IIS- 1254206 and Okawa Foundation Research Award. The views and conclusions are those of the authors and should not be interpreted as representing the official policies of the funding agency, or the U.S. Government.
\vspace{-0.1in}


%\begin{small}
\bibliography{nips2014}
\bibliographystyle{abbrv}
%\end{small}
\newpage
%\section{Appendix A.1}
%\label{sec:admm_proof} 
%% Proof of theorems 
\subsection*{Proof of Theorem \ref{thm:OverlappedBound}  }
In order to be consist with existing work, we define the linear operator $\Phi(\W)= [X_1\w_1, \cdots, X_t\w_t, X_T\w_t]$, where $\w_t$ is the $t$ th column of $\W_{(1)}$ and $X_t$ is the $t$ th slice of tensor $\X$.
\setcounter{equation}{0}
Assume $\hat{\W}$, $\Wstar $ can be decomposed into $\hat{\W}= \frac{1}{N} \sum_{n=1}^N\hat{\W}^n $ and $\Wstar =  \frac{1}{N} \sum_{n=1}^N\Wstar^n$. Let $\Delta = \hat{\W}-\Wstar $, $\Delta^n  = \hat{\W}^n - \Wstar^n$, and $r_n$ be the mode-n  rank of the $\hat{\W}^n$.  Denote
 $L(\W) = \| Y- \Phi(\W)\|_F^2$
 
\begin{lemma} (Adapted from  \cite{tomioka2010estimation} )
\label{lma:lowerbound}
Let $\Delta$ and $\Delta^n$ be the estimation error accordingly. Assume the incoherence assumption holds, which is $\|\Wstar^l_{(n)}\|_{\text{op}} \leq \alpha \quad \forall l\neq n $. We have
\begin{equation*}
\frac{1}{2N^2}\sum_n \|\Delta^n\|_F^2 \leq \frac{1}{2}\|\Delta\|^2_F +\frac{1}{N^2} \alpha (N-1)\sum_n \|\Delta_{(n)}^n\|_*
\end{equation*}
\end{lemma}

\begin{lemma} \cite{agarwal2012noisy}
\label{lma:decomposition}
For model  $\Y = \X(\W )$, if operator $\X$ satisfies restricted strong convexity (RSC) defined in \ref{dfn:RSC}, there is a decomposition of $\Delta_{(n)}$ into two matrices $\Delta_{(n)}= \Delta^A_{(n)} + \Delta^B_{(n)}$, such that $(\Delta^{A}_{(n)})^T \Delta^B_{(n)} =0$, $\mathrm{rank}(\Delta^A_{(n)}) < 2r$ , and $\sum\limits_n \|\Delta^A_{(n)}\|_* \leq 3 \sum\limits_n \|\Delta^B_{(n)}\|_*$, where $A,B$ denote two orthogonal spaces.
\end{lemma}





According to optimality
\begin{equation}
\label{eqn:optimality}
L(\hat{\W}) +\lambda\sum\limits_n \|\hat{\W}^n_{(n)}\|_* \leq L(\Wstar) +\lambda\sum\limits_n \|\Wstar^{n}_{(n)}\|_*
\end{equation}

Use the fact that $Y = \Phi(\Wstar)+\mathcal{E}$, (\ref{eqn:optimality}) can be written as
\begin{equation}
\|\mathcal{E} - \Phi(\Delta) \|_F^2  + \lambda \sum_n \|\hat{\W}^n_{(n)} \|_* \leq  \|\mathcal{E}\|_F^2\\
+ \lambda \sum_n \left( \| \Wstar^{n}_{(n)}\|_{*} \right)\nonumber \\
\end{equation}
Rearrange the inequality, we get 
\begin{eqnarray*}
\|\Phi(\Delta)\|_F^2 &\leq & \lambda \sum_n \left( \|  \Wstar^n_{(n)}\|_{*}- \|\hat{\W}^n_{(n)} \|_* \right)\nonumber \\
& +& 2\langle\mathcal{E}, \Phi(\Delta) \rangle
\end{eqnarray*}
where $\langle\mathcal{E}, \Phi(\Delta)\rangle$ denotes the summation of vector inner product $\sum_{t} (X_t\Delta_t)'*\mathcal{E}_t $.

With triangle inequality,
\begin{equation}
\|  \Wstar^n_{(n)} \|_*- \|\hat{\W}^n_{(n)} \|_* \leq \| \Wstar^n_{(n)}  - \hat{\W}^n_{(n)} \|_*=\|\Delta^n_{(n)}\|_{*}
\end{equation}

With tensor vectorization and H{\"o}lder's inequality 
\begin{eqnarray}
\langle \mathcal{E}, \Phi(\Delta) \rangle \leq \frac{1}{N}\|\Phi(\mathcal{E})\|_{op} \sum_n  \|\Delta^n_{(n)}\|_*
\end{eqnarray}


Put together, we get
\begin{eqnarray}\label{eqn:phi_upper}
\|\Phi(\Delta)\|_F^2  \leq (\lambda +\frac{2}{N}\|\Phi(\mathcal{E})\|_{op}) \sum_n  \|\Delta^n_{(n)}\|_*
\end{eqnarray}
Here we slightly abuse the symbols by using the same $\mathcal{E}$ for the mode 1 folding of noise matrix.

According to the definition of RSC, we know there exists a $\gamma >0$ such that
\begin{equation}\label{eqn:RSC_condition}
\gamma\|\Delta\|_F^2  \leq \|\Phi(\Delta)\|_F^2
\end{equation}

Combining (\ref{eqn:phi_upper} ) and (\ref{eqn:RSC_condition}) with Lemma \ref{lma:lowerbound}, we have
\begin{eqnarray*}
\frac{\gamma}{N^2}\sum_n \|\Delta^{n}_{(n)} \|_F^2  & \leq  &(\lambda +\frac{2}{N}\|\Phi(\mathcal{E})\|_{op}) \sum_n  \|\Delta^n_{(n)}\|_*\\ &+&\frac{2}{N^2} \alpha (N-1)\sum_n \|\Delta_{(n)}^n\|_*
\end{eqnarray*}


Choose $\lambda \geq \frac{2}{N}\|\Phi(\mathcal{E})\|_{op}+ \frac{2}{N^2}\alpha(N-1)$, we have
\begin{eqnarray}
\label{eqn:upperbound}
\frac{\gamma}{N^2}\sum_n \|\Delta^{n}_{(n)} \|_F^2 \leq 2\lambda  \sum_n\|  \Delta^n_{(n)}\|_* 
\end{eqnarray}

Use the property of RSC in Lemma \ref{lma:decomposition}
\begin{eqnarray}
\| \Delta^n_{(n)}\|_*  & = &\|\Delta^{n,A}_{(n)} + \Delta^{n,B}_{(n)}\|_* \nonumber \\
& = &\|\Delta^{n,A}_{(n)} + 3\Delta^{n,A}_{(n)}\|_* \nonumber\\
 &\leq & 4 \|\Delta^{n,A}_{(n)}\|_*   \leq  4\sqrt{2r_n}\|\Delta^{n}_{(n)} \|_F  
\end{eqnarray}

Put  together, we have 
\begin{eqnarray}
\frac{\gamma}{N^2}\sum_n \|\Delta^{n}_{(n)} \|_F^2 &  \leq &  2\lambda  \sum_n\|  \Delta^n_{(n)}\|_*  \nonumber \\
& \leq &  8\lambda\sum_n\sqrt{2r_n}\|\Delta^{n}_{(n)} \|_F \nonumber\\
& \leq & 8\sqrt{2}\lambda\sqrt{\sum_n r_n} \sqrt{\sum_n \|\Delta^n_{(n)}\|_F^2} \nonumber
\end{eqnarray}
Divide both sides  by  $\sqrt{\sum_n \|\Delta^n_{(n)}\|_F^2} $. We have the error bound.

For the mixture approach, since we are flexible in terms of choosing $\{\hat{\W}^n\}$, we can choose one of the $\hat{\W}^{\accentset{\ast} n} = \W$ and all the others to be zero, with $\accentset{\ast} n$ as the mode index for the minimum mode-n rank of $\W$. Thus the latent approach has an improved error of $O(\min r_n)$ while the overlapped approach is $O( \sum\limits_n r_n)$


\appendix
%\section{Example of tensor unfolding in multitask regression model}
%\label{sec:gfolding}
%Let us consider the simpler case of multitask regression in different domains $Y\in \mathbb{R}^{q\times r}$ and $X\in \mathbb{R}^{p\times r}$ represent the response and predictor variables in different tasks. The parameter tensor $\mathcal{A}\in \mathbb{R}^{q\times p\times r}$ models the relationship between the predictors and the response variables, as follows:
\begin{equation}
Y(i, j) = \sum_{k=1}^{p}\mathcal{A}(i, k, j)X(k, j) + \varepsilon(i, j)
\label{eq:var}
\end{equation}
\paragraph{Unfolding in Mode 1}
\begin{align*}
\mathcal{A}_{(1)}= \left[ \begin{array}{ccccc}
\mathcal{A}(1, 1, 1) & \cdots & \mathcal{A}(1, p, 1) & \cdots &\mathcal{A}(1, p, r)\\
\mathcal{A}(2, 1, 1) & \cdots & \mathcal{A}(2, p, 1) & \cdots &\mathcal{A}(2, p, r)\\
\vdots & \cdots& \vdots&\cdots&\vdots\\
\mathcal{A}(q, 1, 1) & \cdots & \mathcal{A}(q, p, 1) & \cdots &\mathcal{A}(q, p, r)\\ \end{array} \right]
\end{align*}

We can rewrite Eq. (\ref{eq:var})
\begin{equation}
Y(:, j) = \mathcal{A}_{(1)}(:,(p(j-1)+1):pj )X(:, j) + \varepsilon(:, j)
\end{equation}

Thus, we can write the loss function in the compact form
\begin{equation}
\min\left\{\left\|\bm{Y} - \mathcal{A}_{(1)}\bm{X}\right\|_F^2\right\}
\label{eq:greedyFold1}
\end{equation}

\noindent where $\bm{Y} = [Y(:, 1), Y(:, 2), \ldots, Y(:, r)]$ and $\bm{X} = [X(:, 1), X(:, 2), \ldots, X(:, r)]$


\paragraph{Unfolding in Mode 2}
\begin{align*}
\mathcal{A}_{(2)}= \left[ \begin{array}{ccccc}
\mathcal{A}(1, 1, 1) & \cdots & \mathcal{A}(q, 1, 1) & \cdots &\mathcal{A}(q, 1, r)\\
\mathcal{A}(1, 2, 1) & \cdots & \mathcal{A}(q, 2, 1) & \cdots &\mathcal{A}(q, 2, r)\\
\vdots & \cdots& \vdots&\cdots&\vdots\\
\mathcal{A}(1, p, 1) & \cdots & \mathcal{A}(q, p, 1) & \cdots &\mathcal{A}(q, p, r)\\ \end{array} \right]
\end{align*}

\begin{equation}
Y(:, j) = [\mathcal{A}_{(2)}(:, (q(j-1)+1):qj)]^{\top}X(:, j) + \varepsilon(:, j)
\end{equation}

\paragraph{Unfolding in Mode 3}
\begin{align*}
\mathcal{A}_{(3)}= \left[ \begin{array}{ccccc}
\mathcal{A}(1, 1, 1) & \cdots & \mathcal{A}(q, 1, 1) & \cdots &\mathcal{A}(q, p, 1)\\
\mathcal{A}(1, 1, 2) & \cdots & \mathcal{A}(q, 1, 2) & \cdots &\mathcal{A}(q, p, 2)\\
\vdots & \cdots& \vdots&\cdots&\vdots\\
\mathcal{A}(1, 1, r) & \cdots & \mathcal{A}(q, 1, r) & \cdots &\mathcal{A}(q, p, r)\\ \end{array} \right]
\end{align*}
For simplicity of notation, we can permute the columns of $\mathcal{A}_{(3)}$ to write as: 
\begin{align*}
\mathcal{A}'_{(3)}= \left[ \begin{array}{ccccc}
\mathcal{A}(1, 1, 1) & \cdots & \mathcal{A}(1, p, 1) & \cdots &\mathcal{A}(q, p, 1)\\
\mathcal{A}(1, 1, 2) & \cdots & \mathcal{A}(1, p, 2) & \cdots &\mathcal{A}(q, p, 2)\\
\vdots & \cdots& \vdots&\cdots&\vdots\\
\mathcal{A}(1, 1, r) & \cdots & \mathcal{A}(1, p, r) & \cdots &\mathcal{A}(q, p, r)\\ \end{array} \right]
\end{align*}
Now reshape each row of $\mathcal{A}'_{(3)}$ into $q\times p$ matrices: denote the matrix resulting from the $j^{th}$ row as $A'_{(3), j}$. We can rewrite Eq. (\ref{eq:var}) as
\begin{equation}
Y(:, j) = A'_{(3), j}X(:, j) + \varepsilon(:, j)
\end{equation}

\section{Proof of Lemma \ref{lem:rank1opt}}
\label{sec:optRank1}
\begin{proof}

The original problem has the following form:
\begin{equation}
\widehat{A} = \argmin_{A: \mathrm{rank}(A) = 1} \left\{ \left\| Y - AX\right\|_F^2 \right\}
\label{eq:original}
\end{equation}

We can rewrite the optimization problem in Eq. (\ref{eq:original}) as estimation of $\alpha \in \mathbb{R}$, $\mathbf{u} \in \mathbb{R}^{q\times 1}, \|\mathbf{u}\|_2 =1$, and $\mathbf{v} \in \mathbb{R}^{p\times 1}, \|\mathbf{v}\|_2 = 1$ such that:
\begin{equation}
\widehat{\alpha}, \widehat{\mathbf{u}}, \widehat{\mathbf{v}} = \argmin_{\alpha, \mathbf{u}, \mathbf{v}: \|\mathbf{u}\|_2 =1, \|\mathbf{v}\|_2 =1} \left\{ \left\| Y - \alpha\mathbf{u}\mathbf{v}^{\top}X\right\|_F^2 \right\}
\end{equation}

We will minimize the above objective function in three steps: First, minimization in terms of $\alpha$ yields $\widehat{\alpha} = \langle Y, \mathbf{u}\mathbf{v}^{\top}X\rangle/\|\mathbf{u}\mathbf{v}^{\top}X\|_F^2$, where we have assumed that $\mathbf{v}^{\top}X \neq \mathbf{0}$. Hence, we have:
\begin{equation}
\widehat{\mathbf{u}}, \widehat{\mathbf{v}} = \argmax_{\mathbf{u}, \mathbf{v}: \|\mathbf{u}\|_2 =1, \|\mathbf{v}\|_2 =1} \frac{\mathrm{tr}((\mathbf{u}\mathbf{v}^{\top}X)^{\top}Y)^2}{\|\mathbf{u}\mathbf{v}^{\top}X\|_F^2 }
\label{eq:afteralpha}
\end{equation}
The objective function can be rewritten  $\mathrm{tr}\left\{(\mathbf{u}\mathbf{v}^{\top}X)^{\top}Y\right\} = \mathrm{tr}\left\{X^{\top}\mathbf{v}\mathbf{u}^{\top}Y\right\} = \mathrm{tr}\left\{YX^{\top}\mathbf{v}\mathbf{u}^{\top}\right\}$.  
Some algebra work on the denominator yields $\|\mathbf{u}\mathbf{v}^{\top}X\|_F^2 = \mathrm{tr}\left\{(\mathbf{u}\mathbf{v}^{\top}X)^{\top}(\mathbf{u}\mathbf{v}^{\top}X) \right\}  = \mathrm{tr}\left\{ X^{\top}\mathbf{v}\mathbf{u}^{\top}\mathbf{u}\mathbf{v}^{\top}X\right\} = \mathrm{tr}\left\{ X^{\top}\mathbf{v}\mathbf{v}^{\top}X \right\} = \mathbf{v}^{\top}XX^{\top}\mathbf{v}$.  
This implies that the denominator is independent of $\mathbf{u}$ and the optimal value of $\mathbf{u}$ in Eq. (\ref{eq:afteralpha}) is proportional to $YX^{\top}\mathbf{v}$. Hence, we need to first find the optimal value of $\mathbf{v}$ and then compute $\mathbf{u} = (YX^{\top}\mathbf{v})/\|YX^{\top}\mathbf{v}\|_2$. Substitution of the optimal value of $\mathbf{u}$ yields:
\begin{equation}
\widehat{\mathbf{v}} = \argmax_{\mathbf{v}: \|\mathbf{v}\|_2 =1} \frac{\mathbf{v}^{\top}XY^{\top}YX^{\top}\mathbf{v}}{\mathbf{v}^{\top}XX^{\top}\mathbf{v} }
\label{eq:afterU}
\end{equation}

Note that the objective function is bounded and invariant of $\|\mathbf{v}\|_2$, hence the $\|\mathbf{v}\|_2 = 1$ constraint can be relaxed. Now, suppose the value of $\mathbf{v}^{\top}XX^{\top}\mathbf{v}$ for optimal choice of vectors $\mathbf{v}$ is $t$.  We can rewrite the optimization in Eq. (\ref{eq:afterU}) as
\begin{align}
\widehat{\mathbf{v}} &= \argmax_{\mathbf{v}} ~\mathbf{v}^{\top}XY^{\top}YX^{\top}\mathbf{v} \nonumber\\
\mathrm{s.t.} &\qquad \mathbf{v}^{\top}XX^{\top}\mathbf{v} = t
\label{eq:afterRelax}
\end{align}

Using the Lagrangian multipliers method, we can show that there is a value for $\lambda$ such that the solution $\widehat{\mathbf{v}}$ for the dual problem is the optimal solution for Eq. (\ref{eq:afterRelax}). Hence, we need to solve the following optimization problem for $\mathbf{v}$:
\begin{align}
\widehat{\mathbf{v}} & = \argmax_{\mathbf{v}: \|\mathbf{v}\|_2 =1} \left\{\mathbf{v}^{\top}XY^{\top}YX^{\top}\mathbf{v}  - \lambda \mathbf{v}^{\top}XX^{\top}\mathbf{v} \right\}\nonumber\\
 & = \argmax_{\mathbf{v}: \|\mathbf{v}\|_2 =1} \left\{\mathbf{v}^{\top}X(Y^{\top}Y - \lambda I) X^{\top}\mathbf{v}  \right\} \label{eq:finalopt}
\end{align}

Eq. (\ref{eq:finalopt}) implies that $\mathbf{v}$ is the dominant eigenvector of $X(Y^{\top}Y - \lambda I) X^{\top}$. Hence, we are able to find the optimal value of both $\mathbf{u}$ and $\mathbf{v}$ for the given value of $\lambda$. For simplicity of notation, let's define $P \triangleq XX^{\top}$ and $Q \triangleq XY^{\top}YX^{\top}$.
Consider the equations obtained by solving the Lagrangian dual of Eq. (\ref{eq:afterRelax}):
\begin{align}
Q\mathbf{v} &= \lambda P\mathbf{v}\label{eq:boundLamU}\\
\|\mathbf{v}^{\top}X\|_2^2 &= t,\label{eq:boundScaling}\\
 \lambda &\geq 0\label{eq:boundLamL}.
\end{align}

Eq. (\ref{eq:boundLamU}) describes a generalized positive definite eigenvalue problem. Hence, we can select $\lambda_{\max} = \lambda_{1}(Q, P)$ which maximizes the objective function in Eq. (\ref{eq:afterU}). The optimal value of $\mathbf{u}$ can be found by substitution of optimal $\mathbf{v}$ in Eq. (\ref{eq:afteralpha}) and simple algebra yields the result in Lemma \ref{lem:rank1opt}.
\end{proof}

\section{Proof of Theorem \ref{thm:greedy}}
\label{sec:gProof}
Note that intuitively, since our greedy steps are optimal in the first mode, we can see that our bound should be at least as tight as the bound of \cite{Shwartz11}.  Here is the formal proof of Theorem \ref{thm:greedy}.
\begin{proof} Let's denote the loss function at $k^{th}$ step by 
\begin{equation}
\mathcal{L}(\Y, \V, \mathcal{W}_{k}) = \sum_{j=1}^{r}\|\V_{(:, :, j)} - \mathcal{W}(:, :, j)\Y_{(:, :, j)}\|_F^2
\end{equation}
Lines 5--8 of Algorithm \ref{alg:greedy} imply:
\begin{align}
\mathcal{L}(\Y, \V, \mathcal{W}_{k}) - \mathcal{L}(\Y, \V, \W_{k+1})&= \mathcal{L}(\Y, \V, \mathcal{W}_{k}) - \min_{m}\inf_{\mathrm{rank}(B) = 1}\mathcal{L}(\Y, \V,  \W_{(m), k}+B)\nonumber\\
&\geq \mathcal{L}(\Y, \V, \mathcal{W}_{k}) - \inf_{\mathrm{rank}(B) = 1}\mathcal{L}(\Y, \V,  \W_{(1), k}+B)\label{eq:gstep1} 
\end{align}

Let's define $B = \alpha C$ where $\alpha \in \mathbb{R},\mathrm{rank}(C) = 1,$ and $\|C\|_2 = 1$.  We expand the right hand side of Eq. (\ref{eq:gstep1}) and write: 
\begin{align*}
&\mathcal{L}(\Y, \V, \W_{k}) - \mathcal{L}(\Y, \V, \W_{k+1})\geq \sup_{\alpha, C: \mathrm{rank}(C) = 1, \|C\|_2 = 1} 2\alpha\langle C\bm{Y}, \bm{V} - \W_{(1), k}\bm{Y}\rangle - \alpha^2\|C\bm{Y}\|_F^2  ,
\end{align*}

\noindent where $\bm{Y}$ and $\bm{V}$ are used for denoting the matrix created by repeating $\Y_{(:, :, j)}$ and $\V_{(:, :, j)}$ on the diagonal blocks of a block diagonal matrix, respectively. Since the algorithm finds the optimal $B$, we can maximize it with respect to $\alpha$ which yields:
\begin{align*}
\mathcal{L}(\Y, \V, \W_{k}) - \mathcal{L}(\Y, \V, \W_{k+1}) &\geq \sup_{C: \mathrm{rank}(C) = 1, \|C\|_2 = 1} \frac{\langle C\bm{Y}, \bm{V} - \W_{(1), k}\bm{Y}\rangle^2}{\|C\bm{Y}\|_F^2}\\
&\geq\sup_{C: \mathrm{rank}(C) = 1, \|C\|_2 = 1} \frac{1}{\sigma_{\max}(\bm{Y})^2}\langle C\bm{Y}, \bm{V} - \W_{(1), k}\bm{Y}\rangle^2  \\
&=\sup_{C: \mathrm{rank}(C) = 1, \|C\|_2 = 1} \frac{1}{\sigma_{\max}(\bm{Y})^2}\langle C, (\bm{V} - \W_{(1), k}\bm{Y})\bm{Y}^{\top}\rangle^2\\
&  =\frac{\sigma_{\max}\left((\bm{V} - \W_{(1), k}\bm{Y})\bm{Y}^{\top} \right)^2}{\sigma_{\max}(\bm{V})}\\
\end{align*}
Define the residual $R_k = \mathcal{L}(\Y, \V, \W_{k}) - \mathcal{L}(\Y, \V, \W^{*})$. Note that $-(\bm{V} - \W_{(1), k}\bm{Y})\bm{Y}^{\top}$ is the gradient of the residual function with respect to $\W_{(1), k}$. Since the operator norm and the nuclear norms are dual of each other, using the properties of dual norms we can write for any two matrices $A$ and $B$
\begin{equation}
\langle A, B \rangle \leq \|A\|_2\|B\|_{*}
\end{equation}
Thus, using the convexity of the residual function, we can show that
\begin{align}
R_k - R_{k+1} &\geq \frac{\left(\left\| \nabla_{\W_{(1), k}}R_k\right\|_2 \|\W_{(1), k} - \W_{(1)}^{*}\|_*\right)^2}{\sigma_{\max}(\bm{Y})^2\|\W_{(1), k} - \W_{(1)}^{*}\|_*^2}\\
& \geq\frac{R_k^2}{\sigma_{\max}(\bm{Y})^2\|\W_{(1), k} - \W_{(1)}^{*}\|_*^2}\label{eq:greedyDecreasing}\\
& \geq \frac{R_k^2}{\sigma_{\max}(\bm{Y})^2\|\W_{(1)}^{*2}\|_*^2}\label{eq:seq}
\end{align}

The sequence in Eq. (\ref{eq:seq}) converges to zero according to the following rate \cite[Lemma B.2]{ShalevShwartz2010}%\citep[Lemma B.2]{ShalevShwartz2010}:
\begin{equation*}
R_k \leq \frac{(\sigma_{\max}(\bm{Y})\|\W_{(1)}^*\|_{*})^2}{(k+1)}
\end{equation*}

The step in Eq. (\ref{eq:greedyDecreasing}) is due to the fact that the parameter estimation error decreases as the algorithm progresses. This can be seen by noting that the minimum eigenvalue assumption ensures strong convexity of the loss function. 
\end{proof}


\section{Convex relaxation with ADMM}
\label{sec:admm_algo}
% ADMM methods and Lemma for Latent 

A convex relaxation approach replaces the constraint $\text{rank}(\W_{(n)})$ with its convex hull $\|\W_{(n)}\|_*$.  The mixture regularization in \cite{tomioka2010estimation} assumes that the $N$-mode tensor $\W$ is a mixture of $N$ auxiliary tensors $\{\Z^{n}\}$, i.e., $\W = \sum_{n=1}^{N}\Z^n$. It regularizes the nuclear norm of only the mode-$n$ unfolding for the $n$ th tensor $\Z^{n}$, i.e, $\sum_{n=1}^N \|\Z^n_{(n)}\|_*$. The resulting convex relaxed optimization problem is as follows:

\vspace{-0.3in}
\begin{align}\label{eqn:mixture}
\widehat{\W} = \argmin_{\W}\left\{\loss(\W;\Y, \V )   + \lambda \sum_n^N \|\Z^n_{(n)}\|_*  \quad
\mathrm{s.t.} \quad  \sum\limits_n^N \Z^n = \W\right\}
\end{align} 
\vspace{-0.1in}


We adapt Alternating Direction Methods of Multiplier  (ADMM) \cite{gabay1976dual} for solving the above problem. Due to the coupling of $\{ \Z^n \}$ in the summation, each $\Z^n$ is not directly separable from other $\Z^{n'}$. Thus, we employ the coordinate descent algorithm to sequentially solve $\{\Z^n\}$. Given the augmented Lagrangian of problem as follows, the ADMM-based algorithm is elaborated in Algo. \ref{alg:ADMM}. 

\begin{eqnarray}\label{eqn:AL}
F(\mathcal{\W},\{\mathcal{Z}^n\},\mathcal{C}) =  \loss(\W;\Y, \V )  + \lambda \sum\limits_{n=1}^N\|\mathcal{Z}^n_{(n)}\|_*  +  \frac{\beta}{2}\sum\limits_n \|\mathcal{\W}-\sum\limits_n\mathcal{Z}^n \|_F^2   -  \langle\mathcal{C}, \mathcal{\W}-\sum\limits_{n=1}^N\mathcal{Z}^n\rangle
\end{eqnarray}


%\begin{algorithm}
%   \caption{ADMM for cokriging with mixture regularizer}
%   \label{alg:ADMM}
%\begin{algorithmic}[1]
%   \STATE {\bfseries Input:} data $\X$ with Laplacian matrix $L$, hyper-parameters $\lambda,\beta$.
%   \STATE {\bfseries Output:} $N$ mode tensor $\W$ 
%   \STATE Initialize $\W, \{\Z^{n}\}, \C$.
%   \REPEAT
%   \FOR{variable $m=1$ {\bfseries to} $M$}
%   \STATE $\W_{:,:,m} \leftarrow   (2 \lambda L  + (\beta + N) I)^{-1}(\frac{1}{\lambda}\X_{:,:,m} + \mathcal{C} + \frac{\beta}{N}\sum_{n=1}^N \mathcal{Z}^n) $
%   \ENDFOR
%   \REPEAT
%   \STATE $\Sigma =  N\W_{(n)}-\frac{N}{\beta}\C - \sum_{n'\neq n} \Z^{n'}_{(n')}$
%   \STATE $\Z^n_{(n)} = \mathrm{shrink}\left( \Sigma , \frac{N^2\lambda}{\beta} \right)$
%   \UNTIL{ solution $\{\Z^n\}$ converge}
%   \STATE $\mathcal{C} \leftarrow  \mathcal{C}-\beta(\mathcal{\W}-\frac{1}{N} \sum\limits_{n=1}^N Z^{n})$
%   
%   \UNTIL{objective function converges}
%\end{algorithmic}
%\end{algorithm}

\begin{algorithm}[h]
 \caption{ADMM for solving Eq. (\ref{eq:greedyUnified})}
 \label{alg:ADMM}
\begin{algorithmic}[1]
   \STATE {\bfseries Input:}
   transformed data $\Y, \V$ of $M$ variables, hyper-parameters $\lambda,\beta$.
   \STATE {\bfseries Output:} $N$ mode tensor $\W$ 
   \STATE Initialize $\W, \{\Z^{n}\}, \C$ to zero.
   \REPEAT
   \STATE $\W \leftarrow   \argmin_{\W}\left\{\loss(\W;\Y, \V ) +\frac{\beta}{2}\|\W - \sum_{n=1}^{N}\Z^{n} - \C  \|_F^2 \right\}$.
   \REPEAT
   \FOR{variable $n=1$ {\bfseries to} $N$}
   \STATE $\Z^n_{(n)} = \mathrm{shrink}_{\frac{\lambda}{\beta}}\left( \W_{(n)}-\frac{1}{\beta}\C - \sum_{n'\neq n} \Z^{n'}_{(n')}  \right)$.
   \ENDFOR
   \UNTIL{ solution $\{\Z^n\}$ converge}
   \STATE $\mathcal{C} \leftarrow  \mathcal{C}-\beta(\mathcal{\W}- \sum\limits_{n=1}^N \Z^{n})$.
   
   \UNTIL{objective function converges}
\end{algorithmic}
\end{algorithm}


%For forecasting, we replace steps 5-6 in Algo. \ref{alg:ADMM} with a gradient descend step for solving $\W_{:,:,m}$ as the resulting close form solution is in the form of Sylverster equation and is expensive to solve. 
The sub-routine $\mathrm{shrink}_{\alpha} (A) $ applies a soft-thresholding rule at level $\alpha$ to the singular values of the input matrix  $A$. The following lemma shows the convergence of ADMM-based solver for our problem.

\begin{lemma} \cite{bertsekas1989parallel}
\label{lem:admm_opt}
For the constrained problem $\min\limits_{x,y} f(x)+g(y), \mathrm{s.t} \quad x \in C_x, y\in C_y, Gx=y$, If either $\{C_x, C_y\}$ are bounded or $G'G$ is invertible, and the optimal solution set is nonempty. A sequence of solutions $\{x, y\}$ generated by ADMM is bounded and every limit point is an optimal solution of the original problem.
\end{lemma}
%\begin{theorem}
%If the optimal solution set of problem (\ref{eqn:mixture})  is nonempty, algorithm \ref{alg:ADMM} generates a bounded sequence of $\{\W,\Z^n, \mathcal{C}\}$ and every limit point of $\W$ is an optimal solution. 
%\end{theorem}
%\proof It easily follows that problem (\ref{eqn:mixture}) satisfied invertible condition for $\W$. With Lemma \ref{lem:admm_opt}, we can reach the conclusion.




\section{Derivation of the unified formulation}
\label{sec:derive}
In this section, we demonstrate how we can use Eq. (\ref{eq:greedyUnified}) to solve Eqs. (\ref{eqn:cokriging_reformulate}) and (\ref{eqn:forecasting_reformulate}). In the cokriging problem, it is easy to see that with $\Y_{:,:,m} = H$ and $\V_{:,:,m} = \X_{\Omega, m}$ for $m = 1, \ldots, M$ the problems are equivalent. In the forecasting problem, $H$ is full rank and the mapping defined by  $\W \mapsto \tilde{\W}: \tilde{\W}_{:,:,m} = H\W_{:,:,m}$ for $m=1, \ldots, M$ preserves the tensor rank, i.e., $\mathrm{rank}(\W) = \mathrm{rank}(\tilde{\W})$. This suggests that we can solve Eq. (\ref{eqn:cokriging_reformulate}) as follows: first solve Eq. (\ref{eq:greedyUnified}) with $\Y_{:,:,m} = \mathbf{X}_{K+1:T, m}$ and $\V_{:,:,m} = \X_{:,:,m}$ and obtain its solution as $\tilde{\W}$; then compute $\W_{:,:,m} = H^{-1}\tilde{\W}_{:,:,m}$.


\end{document}
